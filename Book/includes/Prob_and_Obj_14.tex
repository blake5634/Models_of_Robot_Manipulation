% Problem Statement and Learning Objectives for Chapter 14
 Performing a manipulation task by remote 
control is most often practical when three conditions are met:
\begin{enumerate}
\item The task, or travel to the site of the task, is dangerous or involves
changing size scale. 
\item The task requires significant human skill and is thus beyond the state of 
autonomous robotics.
\item The task is high-value and thus may justify equipment costs.
\end{enumerate}
This chapter concerns analysis and architecture definition for 
remote control of robot manipulators.  


\subsection*{Historical Contextual Note:}
Since initial work in the 1940's, the literature on teleoperation has almost universally 
used the terms ``Master" and ``Slave" denote the user interface device and remote manipulator
respectively.   This terminology negatively impacts readers who have been left with 
the legacy of slavery in their heritage, primarily Black, Indiginous, and People of Color (BIPOC) 
readers.   This text is adopting the proposed terms ``Leader" and "Follower" manipulators 
in the hope of making all readers comfortable with this material. 




After completing this Chapter, students will be able to 
\begin{itemize}

\item 
Define and explain basic approaches to remote control 
of mechanisms (joint-rate control, etc.) and give example
real-world applications. 

\item 
Draw block diagrams and give kinematic equations which
define the basic approaches.

\item 
Explain the concept of leader-follower teleoperation and which types of teleoperation systems can be controlled by 
application of individual decoupled controllers to corresponding joint pairs. 

\item 
Define Scaling in teleoperation and give some kinematic 
equations to implement it in one or more example architectures.

\item 
Define indexing in teleoperation and give transform diagrams and equations to illustrate how indexing is used. 

\item 
Identify potential problems with implementation of indexing
and scaling in practical teloperation systems and common
work-arounds. 


\end{itemize}





