%% Default Latex document template
%%
%%  blake@rcs.ee.washington.edu

\documentclass{article}

% Uncomment for bibliog.
%\bibliographystyle{unsrt}

\usepackage{graphicx}
\usepackage{amsmath}

%\usepackage{lineno}
%\linenumbers
%%%%%%%%%%%%%%%%%%%%%%%%%%%%%%%%%%%%%%%%5
%
%  Set Up Margins

%%%%%%%%%%%%%%%%%%%%%%%%%%%%%%%%%%%%%%%%%%%%%%%%%
% include file for:
%      Critical Page setup dimensions
%            DO NOT MODIFY
%       (for help see "Latex Line by Line" p 260)
%
\setlength\oddsidemargin{0in}
\setlength\evensidemargin{0in}

\usepackage[left=0.98in, right=0.98in, top=1.0in, bottom=1.0in]{geometry}

% %Top Margin and header
% \setlength\voffset{-0.94in}
% \setlength\topmargin{0.25in}
% \setlength\headheight{0.25in}
% %\setlength\headwidth{6.5in}
% \setlength\headsep{0.25in}
% %Body
% \setlength\textwidth{6.5in}
% \setlength\textheight{9.50in}
% %Footer
% %\setlength\footheight{0.5in}
% \setlength\footskip{0.3750in}
% Line spacing for 6 lines per inch
\linespread{0.894}  % 1.0 = single    1.6 = double
%
%          END of Critical Page Setup Dimensions
%%%%%%%%%%%%%%%%%%%%%%%%%%%%%%%%%%%%%%%%%%%%%%%%%%%

%%%%%%%%%%%%%%%%%%%%%%%%%%%%%%%%%%%%%%%%%%%%%%%%%%%
%
% Useful style and math macros
%


\newcommand\Dfrac[2]{\frac{\displaystyle #1}{\displaystyle #2}}
\newcommand\beq{\begin{equation}}
\newcommand\eeq{\end{equation}}

\newcommand\bmat{\begin{bmatrix}}
\newcommand\emat{\end{bmatrix}}

\newenvironment{solution}
{\vspace{0.125in} {\bf SOLUTION:} \\ }
{\vspace{0.25in}}





%%%%%%%%%%%%%%%%%%%%%%%%%%%%%%%%%%%%%%%%%%%%%%%%%
%
%         Page format Mods HERE
%
%Mod's to page size for this document
\addtolength\textwidth{0cm}
\addtolength\oddsidemargin{0cm}
\addtolength\headsep{0cm}
\addtolength\textheight{0cm}
%\linespread{0.894}   % 0.894 = 6 lines per inch, 1 = "single",  1.6 = "double"

%\lhead{LEFT HEADER}
%\chead{CENTER HEADER}
%\rhead{RIGHT HEADER}
%\lfoot{Hannaford, U. of Washington}
%\rfoot{\today}
%\cfoot{\thepage}
\begin{document}

\title{EE546 HW 2 \\ University of Washington}
\setcounter{section}{2}


\maketitle

\subsection{}          % 1.1
Find the rotation matrix described by
\[
q_3 = q_1\cdot q_2
\]
where
\[
q_1 = \bmat 9.76\\.096\\-.0512\\-.187\emat \qquad q_2 = \bmat .998\\.00309\\.0604\\-.00810 \emat
\]

\subsection{}

Find the unit quaternion corresponding to the following sequence of rotations
\begin{enumerate}
  \item rotate about the vector $[0.7\; 5\; -.3 ]^T$ by $30^\circ$
  \item rotate about the current $\hat{y}$  by $16^\circ$
  \item rotate about the vector $[10\;-1.6\;2 ]^T$ by $23^\circ$ in hte original (unrotated) frame.
\end{enumerate}

\subsection{}

A car has a roof-attached frame like the car of Example 2.6 (course notes).   The car
\begin{enumerate}
    \item drives in reverse for 10m
    \item turns left by $20^\circ$
    \item drives forward by 20m
    \item turns right by $90^\circ$
    \item goes down a ramp at angle $10^\circ$ for 10m
\end{enumerate}

What is the 4x4 homogenenous transform which represents the position and orientation of the car's frame after these moves?    Assume that all turns can be accomplished without any translation (unlike a real car).


\subsection{}
A can rolls on a table:

\includegraphics[width=5.0in]{00861.png}

The following facts are known:

\begin{itemize}
  \item The table is 1m high
  \item Frame T is rotated by $-30^\circ$ about $Z_W$
  \item The $X_T, \; Y_T$ coordinates of the origin for $\{C\}$ are
  \[
  {^TO_C} = \bmat 7\;3\;? \emat
  \]
  \item The radius of the can is 10cm
  \item ${^TX_C} = [7.707\; 3.707\; ?]$
  \item $Z_C$ makes an angle of $60^\circ$ with the tabletop

    \includegraphics[width=45mm]{00862.png}

\end{itemize}

Find the configuration of the can in the world:
\[
{^W_CT}
\]


\subsection{}

\includegraphics[width=6.5in]{00863.png}

A camera estimates the position and orientation of a puck on a table.  Unfortunately, the camera mount is flexible and is held together by duct tape so we do not
precisely know the camera location.  The camera can get the position and orientation of a special marker (consisting of three dots) which is mounted on to the table in a known position and orientation.

The following relationships are known
\[
{^W_TT}\quad { ^T_MT}\quad  {^B_ET}\quad  {^W_BT}\quad  {^C_MT}\quad  {^C_{P1}T}\:
\]

Draw the transform graph and find the configuration of the puck in frame $B$, ${^B_{P1}T}$



\subsection{}

\includegraphics[width=135mm]{00864.png}

\begin{itemize}
    \item The point $^0P$ is $[9\; 4\; 0]^T$
    \item The sides of the box are parallel to Frame 0.
\end{itemize}

Find
\[
{^0_BT}
\]

\end{document}

